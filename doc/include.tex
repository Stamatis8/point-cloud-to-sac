\usepackage[utf8]{inputenc}

\usepackage{float}
\usepackage{graphicx}
\usepackage{caption}
\usepackage{subcaption}
\usepackage{amsfonts}
\usepackage[dvipsnames]{xcolor}

\usepackage{svg}
\usepackage{amsmath}
\usepackage{amsthm}
\usepackage{amssymb}
\usepackage{mathabx}

\usepackage{ulem}

\usepackage{cancel}

\usepackage{listings}
\usepackage{hyperref}
\hypersetup{
    colorlinks=true,
    allcolors=blue,
    }

\usepackage{geometry}
 \geometry{
 a4paper,
 left=15mm,
 top=20mm,
 right = 15mm,
 bottom = 20mm
 }

\usepackage{fancyhdr}
\setlength{\headheight}{15.2pt}
\pagestyle{fancy}
\fancyhead[RO]{\chaptermark}
\fancyfoot{}
\fancyfoot[R]{Page \thepage}

\usepackage[style=authoryear-ibid,backend=biber]{biblatex}
\addbibresource{bibliography.bib}

\theoremstyle{definition}
\newtheorem{definition}{Definition}[section]
\newtheorem{example}{Example}[section]
\newtheorem{exercise}{Exercise}[section]
\newtheorem{proposition}{Proposition}[section]
\newtheorem{theorem}{Theorem}[section]
\newtheorem{lemma}{Lemma}[section]
\newtheorem{axiom}{Axiom}
\newtheorem{corollary}{Corollary}[section]
\theoremstyle{remark}
\newtheorem{remark}{Remark}[section]
\newtheorem{question}{Question}[section]
\newtheorem{answer}{Answer}[section]

\usepackage{subfiles}

\usepackage{changepage}

\usepackage{framed}

\newcommand{\AND}{\wedge}
\newcommand{\OR}{\vee}
\newcommand{\NOT}{\neg}
\newcommand{\EMPTY}{\varnothing}

\newcommand{\UNDER}{\uline}

\newcommand{\newpar}{\vspace{\baselineskip}\par\noindent}